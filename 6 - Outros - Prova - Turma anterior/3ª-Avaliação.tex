\documentclass[]{article}
\usepackage{lmodern}
\usepackage{amssymb,amsmath}
\usepackage{ifxetex,ifluatex}
\usepackage{fixltx2e} % provides \textsubscript
\ifnum 0\ifxetex 1\fi\ifluatex 1\fi=0 % if pdftex
  \usepackage[T1]{fontenc}
  \usepackage[utf8]{inputenc}
\else % if luatex or xelatex
  \ifxetex
    \usepackage{mathspec}
  \else
    \usepackage{fontspec}
  \fi
  \defaultfontfeatures{Ligatures=TeX,Scale=MatchLowercase}
\fi
% use upquote if available, for straight quotes in verbatim environments
\IfFileExists{upquote.sty}{\usepackage{upquote}}{}
% use microtype if available
\IfFileExists{microtype.sty}{%
\usepackage{microtype}
\UseMicrotypeSet[protrusion]{basicmath} % disable protrusion for tt fonts
}{}
\usepackage[margin=1in]{geometry}
\usepackage{hyperref}
\hypersetup{unicode=true,
            pdftitle={Trabalho - 3ª Avaliação},
            pdfauthor={Filipe Coelho de Lima Duarte},
            pdfborder={0 0 0},
            breaklinks=true}
\urlstyle{same}  % don't use monospace font for urls
\usepackage{graphicx,grffile}
\makeatletter
\def\maxwidth{\ifdim\Gin@nat@width>\linewidth\linewidth\else\Gin@nat@width\fi}
\def\maxheight{\ifdim\Gin@nat@height>\textheight\textheight\else\Gin@nat@height\fi}
\makeatother
% Scale images if necessary, so that they will not overflow the page
% margins by default, and it is still possible to overwrite the defaults
% using explicit options in \includegraphics[width, height, ...]{}
\setkeys{Gin}{width=\maxwidth,height=\maxheight,keepaspectratio}
\IfFileExists{parskip.sty}{%
\usepackage{parskip}
}{% else
\setlength{\parindent}{0pt}
\setlength{\parskip}{6pt plus 2pt minus 1pt}
}
\setlength{\emergencystretch}{3em}  % prevent overfull lines
\providecommand{\tightlist}{%
  \setlength{\itemsep}{0pt}\setlength{\parskip}{0pt}}
\setcounter{secnumdepth}{0}
% Redefines (sub)paragraphs to behave more like sections
\ifx\paragraph\undefined\else
\let\oldparagraph\paragraph
\renewcommand{\paragraph}[1]{\oldparagraph{#1}\mbox{}}
\fi
\ifx\subparagraph\undefined\else
\let\oldsubparagraph\subparagraph
\renewcommand{\subparagraph}[1]{\oldsubparagraph{#1}\mbox{}}
\fi

%%% Use protect on footnotes to avoid problems with footnotes in titles
\let\rmarkdownfootnote\footnote%
\def\footnote{\protect\rmarkdownfootnote}

%%% Change title format to be more compact
\usepackage{titling}

% Create subtitle command for use in maketitle
\providecommand{\subtitle}[1]{
  \posttitle{
    \begin{center}\large#1\end{center}
    }
}

\setlength{\droptitle}{-2em}

  \title{Trabalho - 3ª Avaliação}
    \pretitle{\vspace{\droptitle}\centering\huge}
  \posttitle{\par}
    \author{Filipe Coelho de Lima Duarte}
    \preauthor{\centering\large\emph}
  \postauthor{\par}
    \date{}
    \predate{}\postdate{}
  

\begin{document}
\maketitle

\hypertarget{projeto-da-disciplina-estagio-supervisionado-ii-do-semestre-2019.1}{%
\subsection{Projeto da disciplina Estágio Supervisionado II do semestre
2019.1}\label{projeto-da-disciplina-estagio-supervisionado-ii-do-semestre-2019.1}}

O trabalho consistirá em um projeto em que cada aluno possuirá uma base
de dados diferente.

O projeto é dividido em duas etapas:

\hypertarget{considere-os-dados-referente-ao-seu-nome-contidos-na-pasta.-por-exemplo-antonio-roberto-utilizara-o-arquivo-denominado-base_antonio.csv.}{%
\subsubsection{1. Considere os dados referente ao seu nome contidos na
pasta. Por exemplo, Antonio Roberto utilizará o arquivo denominado
``base\_antonio.csv''.}\label{considere-os-dados-referente-ao-seu-nome-contidos-na-pasta.-por-exemplo-antonio-roberto-utilizara-o-arquivo-denominado-base_antonio.csv.}}

\begin{enumerate}
\def\labelenumi{\alph{enumi})}
\item
  Faça as estatísticas descritivas da variável ``valor'' que se refere
  aos sinistros.
\item
  Plote um histograma para a variável ``valor''.
\item
  Estime os parâmetros pelo método da máxima verossimilhança para as
  distribuições: ``normal'', ``gamma'', ``lognormal'', ``weibull'',
  ``pareto''.
\item
  Apresente os gráficos dos modelos probabilísticos estimados em relação
  à distribuição empírica.
\item
  Indique qual o modelo que melhor se ajusta aos dados dos sinistros e
  explique o porquê.
\end{enumerate}

\hypertarget{calcule-o-premio-puro-pelo-principio-do-valor-esperado-para-a-carteira-pelo-modelo-de-risco-coletivo.-utilize-o-carregamento-de-seguranca-theta-10.}{%
\subsubsection{\texorpdfstring{2. Calcule o Prêmio Puro pelo Princípio
do Valor Esperado para a Carteira pelo Modelo de Risco Coletivo. Utilize
o carregamento de segurança
\(\theta = 10\)\%.}{2. Calcule o Prêmio Puro pelo Princípio do Valor Esperado para a Carteira pelo Modelo de Risco Coletivo. Utilize o carregamento de segurança \textbackslash{}theta = 10\%.}}\label{calcule-o-premio-puro-pelo-principio-do-valor-esperado-para-a-carteira-pelo-modelo-de-risco-coletivo.-utilize-o-carregamento-de-seguranca-theta-10.}}

\begin{itemize}
\tightlist
\item
  Adrilayne
\end{itemize}

Uitilzará uma base de dados que corresponde aos sinistros de uma
carteira de seguro saúde. A distribuição referente à frequência de
sinistros \(N\) será a Binomial Negativa com parâmetros \(r=3\) e
\(p=0.9\) A quantidade de segurados na carteira é 10000.

\begin{itemize}
\tightlist
\item
  Antonio Roberto
\end{itemize}

Utilizará uma base de dados que corresponde aos sinistros de uma
carteira de automóvel. A distribuição referente à frequência de
sinistros por apólice será uma Binomial Negativa com parâmetros \(r=2\)
e \(p=0.85\). A quantidade de segurados na carteira é 5000.

\begin{itemize}
\tightlist
\item
  José de Matos
\end{itemize}

Utilizará uma base de dados que corresponde aos sinistros de uma
carteira de seguro de incêndio. A distribuição referente à frequência de
sinistros por apólice será uma Binomial Negativa com parâmetros \(r=2\)
e \(p=0.98\). A quantidade de segurados na carteira é 1000.

\begin{itemize}
\tightlist
\item
  Lucas
\end{itemize}

Utilizará uma base de dados que corresponde aos sinistros de uma
carteira de seguro de responsabilidade civil. A distribuição referente à
frequência de sinistros por apólice é uma Poisson com parâmetro
\(\lambda = 1\). A quantidade de segurados na carteira é 600.

\begin{itemize}
\tightlist
\item
  Ludmila
\end{itemize}

Utilizará uma base de dados que corresponde aos sinistros de uma
carteira de seguro viagem. A distribuição referente à frequêcia de
sinistros por apólice será uma Poisson com parâmetro \(\lambda = 2\). A
quantidade de segurados na carteira é 5000.

\hypertarget{calcule-o-premio-comercial-com-carregamento-para-despesas-alpha-18.}{%
\subsubsection{\texorpdfstring{3. Calcule o Prêmio Comercial com
carregamento para despesas
\(\alpha = 18\)\%.}{3. Calcule o Prêmio Comercial com carregamento para despesas \textbackslash{}alpha = 18\%.}}\label{calcule-o-premio-comercial-com-carregamento-para-despesas-alpha-18.}}

\hypertarget{elabore-2-contratos-de-resseguro-para-mitigar-os-riscos-da-carteira-de-sua-seguradora.}{%
\subsubsection{4. Elabore 2 contratos de Resseguro para mitigar os
riscos da carteira de sua
seguradora.}\label{elabore-2-contratos-de-resseguro-para-mitigar-os-riscos-da-carteira-de-sua-seguradora.}}

O primeiro contrato será proporcional e o segundo não proporcional. Além
disso, para cada tipo de contrato, você terá que elaborar 3 cenários
para cada contrato de Resseguro e apresentar:

\begin{enumerate}
\def\labelenumi{\alph{enumi})}
\item
  Prêmio Puro e Comercial Retido
\item
  Prêmio Puro e Comercial da Resseguradora considerando o carregamento
  para despesas da resseguradora de 20\%.
\item
  Sinistro Retido
\item
  Sinistro de responsabilidade da Resseguradora
\item
  Probabilidade dos Sinistros Retidos Superarem o Prêmio Puro Retido.
\end{enumerate}

\hypertarget{calcular-a-partir-do-quesito-anterior---contrato-de-resseguro-a-reserva-de-risco-mu-que-garanta-a-solvencia-da-seguradora-em-95-ao-longo-de-1-ano.}{%
\subsubsection{\texorpdfstring{5. Calcular, a partir do quesito anterior
- Contrato de Resseguro, a Reserva de Risco (\(\mu\)) que garanta a
Solvência da Seguradora em 95\% ao longo de 1
ano.}{5. Calcular, a partir do quesito anterior - Contrato de Resseguro, a Reserva de Risco (\textbackslash{}mu) que garanta a Solvência da Seguradora em 95\% ao longo de 1 ano.}}\label{calcular-a-partir-do-quesito-anterior---contrato-de-resseguro-a-reserva-de-risco-mu-que-garanta-a-solvencia-da-seguradora-em-95-ao-longo-de-1-ano.}}


\end{document}
